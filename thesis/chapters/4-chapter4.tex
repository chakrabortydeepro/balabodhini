\chapter{Bhaṭṭa Jagaddhara and Śitikaṇṭha}


\section{Jagaddhara and Śitikaṇṭha's Time}

Jagaddhara made no mention of his date in his works. However, Śitikaṇṭha clearly indicated the ruler of Kashmir at the time he composed the \emph{nyāsa} and specified the exact date of completing the first half of the text (\emph{pūrvārdha}). Moreover, he states his familial relationship with Jagaddhara, which assists us in determining Jagaddhara's date.

Śitikaṇṭha initiated his \emph{nyāsa} with 11 preliminary verses, concluding with two panegyric verses dedicated to Sultān Ḥasan Šāh, the son of Haydar Šāh, under whose reign he composed the \emph{nyāsa}.

\begin{quote}	
	\textsanskrit{grāme grāme'grahārān maṭhadharaṇiyutān karmaṭhebhyaḥ kaṭhebhyaḥ 
	sabhyebhyo yo vyatārīdripugurunagarīryo garīyānabhaitsīt|\\
	akṣāṇyakṣīṇaśaktirvyavahṛticaturo yo vyajaiṣīcca tasmin
	bhūjānau \textbf{hassanā}khye bhuvamavati mayā tanyate grantha eṣaḥ|| 10||\\
	sarvakṣmāpatimaulirāhitanavaprājyasvarājyaḥ paraṃ
	tattvātattvavicārakāridhiṣaṇo gāmbhīryaśauryānvitaḥ |\\	
	kaśmīrādhipatiḥ kṛpājalanidhirvikhyātakīrtiściraṃ
	jīyād \textbf{haidaraśāhisūnu}ranaghaḥ kandarpadarpāpahaḥ || 11 ||}
	(BBP 1.1.1.1.10--1)
	

	“I am crafting this book during the reign of the king called \textbf{Hassana (Ḥasan)}, who bestowed royal land grants with monasteries in numerous villages to efficient courtiers, who were Brahmins of the Vedic Kaṭha school. He, the mighty one, shattered the big cities of his enemies. With ingenuity in his demeanor, he triumphed over his senses, wielding unwaning power. He stands as the pinnacle among all monarchs, ruling over his expansive new kingdom. Utmost in intelligence, he discerns between real and unreal, and is endowed with both dignity and valour. May the ruler of Kashmir, the faultless son of \textbf{Haidaraśāhi (Ḥaydar Šāh)}, an ocean of compassion, renowned for his fame, the vanquisher of Cupid's pride, live long.”	
\end{quote}

It is clear from this reference that Śitikaṇṭha began composing his work when Ḥasan Šāh (\emph{r.} 1472--1484)\footcite[109]{Hasan2024} was ruling. 

Śitikaṇṭha mentioned the exact date of the completion of the \emph{Nāmaprakaraṇa} (2nd book of \emph{Kātantra}) at the end of the first half (\emph{pūrvārdha}) of his \emph{nyāsa}---

\begin{quote}
	

\textsanskrit{tryaṅkaviśvamite śāke varṣe muniyugairmite|\\
nabhasyasitasaptamyāṃ samāpto nāmanirṇayaḥ||}\footnote{\gls{B3} f.224v.}

“The \emph{Nāmaprakaraṇa} is completed in the (expired) Śaka year (1393) counted by three, number (=9), and \emph{Viśva} (=13) and in the Laukika year (47) counted by \emph{muni} (=7) and \emph{yuga} (4), on the seventh day of the waxing lunar fortnight (\emph{sitasaptamī}) in the month of Nabhasya (Bhādrapada).”

\end{quote}
The Laukika era, also known as the Śaptarṣi or Śāstra era, stands out as the most prevalent dating system used in the Sharada manuscripts. A significant limitation of this dating system lies in its omission of centuries from the years indicated, restarting at 0, once the count reaches 100 years. The Laukika year mentioned in manuscripts is usually the current year, while the Śaka year is most often the expired year. In this context, the Śaka year 1393 represents the expired year. The current Laukika year 47 corresponds to the current Śaka year 1394. Adhering to the Āmanta scheme, the seventh day of the waxing lunar fortnight in the month of Bhādrapada of Śaka 1394 equates to Tuesday August 20, 1472.\footnote{For the conversion, I used the free web 			
	application HIC (\url{https://hic.efeo.fr/}), originally authored by L. Gislén and J. C. Eade, developed into a web application by Toni Kustiana under the supervision of Arlo Griffiths with the funding furnished by the École française d’Extrême-Orient (EFEO).}
This date coincides with date of Sultan Ḥasan Šāh who ascended the throne after his father Sultan Ḥaydar Šāh's death on April 13, 1472.\footcite[108]{Hasan2024} 

In one of his prefatory verses, Śitikaṇṭha also mentions his familial ties with Bhaṭṭa Jagaddhara---

\begin{quote}

\textsanskrit{yo bālabodhinyabhidhāṃ budhendro jagaddharo yāṃ vitatāna vṛttim|\\
tannaptṛkanyātanayātanūjo vyākhyāmi tāṃ śrīśitikaṇṭhako'lpam|| 8 ||}

“The great intellectual Jagaddhara composed the commentary called the \emph{Bālabodhinī}. I, Śitikaṇṭha, his grandson's daughter's daughter's son, am explaining that modestly.”\footnote{%
	Stainton noted a discrepancy in interpreting the expression “tannaptṛkanyātanayatanujaḥ” (\cite[86–87 n.94]{Stainton2019}, also in \cite[346 n.23]{Stainton2016}): Sanderson identified Śitikaṇṭha as the son of the daughter of Jagaddhara's grandson \parencite[332 n.339]{Sanderson2007}, while Rastogi provided conflicting information within the same book, stating Śitikaṇṭha as the “son of the daughter's daughter of the great-grandson of Jagaddhara” \parencite[223]{Rastogi1979} then on p.?? he showed Śitikaṇṭha as ???? \parencite{}. However, these discrepancies seem to be either typographical errors or the result of carelessness. Rastogi referred to Mahavir Prasad Dwivedi on the same page 223, who accurately described Śitikaṇṭha as the son of Jagaddhara's grandson's daughter's daughter: \texthindi{जगद्धर के नाती की लड़की की लड़की का लड़का था।} \parencite[131]{Dvivedi1928}. On the other hand, Premvallabh Tripathi interprets the word \emph{naptṛ} as daughter's son (\emph{dauhitra}): “\texthindi{जगद्धरके दौहित्रकी दौहित्रीके पुत्र थे।}” \parencite[24]{Panta1964}. However, lexicographers suggest that naptṛ can refer to any grandson, whether from a son or a daughter. See \cite[3964b]{Bhattacarya1962}.	
	} 

\end{quote}
Śitikaṇṭha's lineage places him five generations removed from Jagaddhara. Assuming a generational span of 20 years, we can infer that Jagaddhara flourished during the latter half of the fourteenth century.\footcite[3]{Durgaprasad1891} 

\section{Jagaddhara's Works}

The most popular works by Bhaṭṭa Jagaddhara are the \emph{Stutikusumāñjali} and the \emph{Bālabodhinī}. A significant number of manuscripts of these two texts in various archives prove their popularity. Other than these two texts, Jagaddhara also authored a few other works such as the \emph{Varṇaśikṣāsaṅkṣepa} and perhaps a \emph{citrakāvya}. 

\subsection{Stutikusumāñjali}

The \emph{Stutikusumāñjali} is Jagaddhara’s only published work. It is a collection of hymns dedicated to Śiva. The text spans 38 cantos, comprising a total of 1425 verses, followed by a group of 16 verses in which the poet describes his lineage.\footnote{
	In the 14th verse of the group of 16 concluding verses, Jagaddhara indicates the total number of pādas of the \emph{Stutikusumāñjali} as 4700, suggesting the total number of verses to be 1425. 
	\begin{quote}
		\textsanskrit{nikṣiptaṃ śatasaptakena sahitaṃ pādāyutārdhaṃ mayā...}\parencite{bibid}
	\end{quote}
	In the printed editions, however, we find 1423 verses excluding the group of 16 concluding verses. It is possible that these two missing verses could be identified by consulting additional manuscripts.
	} 
These cantos, termed \emph{Stotra}s, function as independent compositions. Some of these \emph{stotra}s, such as the \emph{Dīnākrandanastotra} or the \emph{Mahāyamakastotra}, also exist independently in distinct manuscripts.

In 1681 AD, Rājānaka Ratnakaṇṭha composed a commentary on the \emph{Stutikusumāñjali} called the Laghupañcikā. 

The \emph{Stutikusumāñjali} was so popular that se verses from this text appeared in later anthologies. For example, Vallabhadeva quotes a number of Jagaddhara's verses in his anthology \emph{Subhāṣitāvali}.\footnote{See \cite[36--37]{Peterson1886}.} Manuscripts of the \emph{Stutikusumāñjali} in both Sharada and Devanagari are found in archives of Jammu and Kashmir as well as of Rajasthan, Maharashtra and West Bengal. A number of these manuscripts are accompanied by Ratnakaṇṭha's \emph{Laghupañcikā}.\footcite[87]{Stainton2019} A number of printed editions and translations also endorse the popularity of this text. 

Pandit Durgaprasad and Kasinath Pandurang Parab produced the first edition of the \emph{Stutikusumāñjali}, accompanied by Ratnakaṇṭha's \emph{Laghupañcikā}, using two manuscripts. The newer manuscript contained both the main text and the commentary, while the older one contained only the main text. This edition appeared in the Kāvyamālā Series of the Nirṇaya Sāgara Press in 1891.\footcite{Durgaprasad1891} Encouraged by an article on the \emph{Stutikusumāñjali} authored by the eminent Hindi writer Mahavir Prasad Dwivedi, Premvallabh Tripathi, in 1938, published a Hindi translation of the text alongside the original taken from the editio princeps. In the subsequent revised edition, Tripathi appended the commentary \emph{Laghupañcikā}.\footcite{Panta1964} Subsequent editions of the \emph{Stutikusumāñjali} rely on either Tripathi's version or the editio princeps of Durgaprasad and Parab. Despite the availability of several manuscripts of the text in various archives, all editions ultimately stem from the two manuscripts originally consulted in the editio princeps. 

Following Mahavir Prasad Dwivedi's work\footcite{Dvivedi1928}, several studies on the \emph{Stutikusumāñjali} have been published.\footnote{For example, \cite{Tirtha1989}} Most of them are in Hindi. A detailed study of the text has been carried out recently by Hamsa Stainton.\footcites{Stainton2016}{Stainton2019}


\subsection{Bālabodhinī}

While this dissertation is perhaps the effort to publish any segment of the \emph{Bālabodhinī}, the text has maintained a significant presence in the Kashmir valley. Alongside the \emph{Laghuvṛtti}, the \emph{Bālabodhinī} stands as one of the important commentaries on the \emph{Kātantra} extensively studied in Kashmir. Although \emph{Laghuvṛtti} surpasses the \emph{Bālabodhinī} in terms of the number of manuscripts found across various archives, the extant manuscripts of the Bālabodhinī are substantial. We are acquainted with over 35 manuscripts, whether complete or fragmented, of the \emph{Bālabodhinī}. Some manuscripts of the \emph{Bālabodhinī} are also available in the Devanagari script.  

Jagaddhara mentioned in one of the prefatory verses of the \emph{Bālabodhinī} that he composed the text for the study of his son Yaśodhara—

\begin{quote}
	
\textsanskrit{svasutasya śiśor yaśodharasya smaraṇārthaṃ vihito mayā śramo'yam|\\
	upayogam iyād yadi prasaṅgād aparatrāpi tato bhaved avandhyaḥ||}

“I made this effort for my own little son Yaśodhara's memorization. if it incidentally finds use elsewhere too it would not be fruitless.”
	
\end{quote}


In terms of the contents, Bālabodhinī appears to be a slightly extended version of the Laghuvṛtti. 


\subsection{Varṇaśikṣāsaṅkṣepa}

\subsection{Citrakāvya}

\subsection{Other Works Attributed to Jagaddhara}

\section{Śitikaṇṭha's Works}

\subsection{Bālabodhinīprakāśa}

\section{Jagaddhara's lineage}

\section{Śitikaṇṭha's lineage}