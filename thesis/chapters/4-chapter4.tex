\chapter{Bhaṭṭa Jagaddhara and Śitikaṇṭha}

This chapter primarily provides a biographical account of Jagaddhara and Śitikaṇṭha, the authors of the texts edited in this thesis. They are both notable Kashmiri authors from the fourteenth and fifteenth centuries, respectively.

\section{Jagaddhara and Śitikaṇṭha's Time}

Jagaddhara made no mention of his date in his works. However, Śitikaṇṭha clearly indicated the ruler of Kashmir at the time he composed the \emph{nyāsa} and specified the exact date of completing the first half of the text (\emph{pūrvārdha}). Moreover, he states his familial relationship with Jagaddhara, which assists us in determining Jagaddhara's date.

Śitikaṇṭha initiated his \emph{nyāsa} with 11 preliminary verses, concluding with praise for Sultan Hassan Shah, the son of Haider Shah, under whose reign he composed the \emph{nyāsa}.

\begin{quote}	
	\textsanskrit{grāme grāme'grahārān maṭhadharaṇiyutān karmaṭhebhyaḥ kaṭhebhyaḥ\\
	sabhyebhyo yo vyatārīdripugurunagarīryo garīyānabhaitsīt|\\
	akṣāṇyakṣīṇaśaktirvyavahṛticaturo yo vyajaiṣīcca tasmin\\
	bhūjānau \textbf{hassanā}khye bhuvamavati mayā tanyate grantha eṣaḥ|| 10||\\
	sarvakṣmāpatimaulirāhitanavaprājyasvarājyaḥ paraṃ\\
	tattvātattvavicārakāridhiṣaṇo gāmbhīryaśauryānvitaḥ |\\	
	kaśmīrādhipatiḥ kṛpājalanidhirvikhyātakīrtiściraṃ\\
	jīyād \textbf{haidaraśāhisūnu}ranaghaḥ kandarpadarpāpahaḥ || 11 ||}
	BBP 1.1.1.1.10--11
	

	“I am crafting this book during the reign of the king called \textbf{Hassana (Hassan)}, who bestowed royal land grants with monasteries in numerous villages to efficient courtiers, who were Brahmins of the Vedic Kaṭha school. He, the mighty one, shattered the big cities of his enemies. With ingenuity in his demeanor, he triumphed over his senses, wielding unwaning power. He stands as the pinnacle among all monarchs, ruling over his expansive new kingdom. Utmost in intelligence, he discerns between real and unreal, and is endowed with both dignity and valour. May the ruler of Kashmir, the faultless son of \textbf{Haidaraśāhi (Haider Shah)}, an ocean of compassion, renowned for his fame, the vanquisher of Cupid's pride, live long.”	
\end{quote}



\gls{B3}

\section{Jagaddhara's Works}

\subsection{Stutikusumāñjali}

\subsection{Bālabodhinī}

\subsection{Varṇaśikṣāsaṅkṣepa}

\subsection{Citrakāvya}

\subsection{Other Works Attributed to Jagaddhara}

\subsection{Śitikaṇṭha's Works}

\subsubsection{Bālabodhinīprakāśa}

