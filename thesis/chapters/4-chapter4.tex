\chapter{Bhaṭṭa Jagaddhara and Śitikaṇṭha}

This chapter primarily provides a biographical account of Jagaddhara and Śitikaṇṭha, the authors of the texts edited in this thesis. They are both notable Kashmiri authors from the fourteenth and fifteenth centuries, respectively.

\section{Jagaddhara and Śitikaṇṭha's Time}

Jagaddhara made no mention of his date in his works. However, Śitikaṇṭha clearly indicated the ruler of Kashmir at the time he composed the \emph{nyāsa} and specified the exact date of completing the first half of the text (\emph{pūrvārdha}). Moreover, he states his familial relationship with Jagaddhara, which assists us in determining Jagaddhara's date.

Śitikaṇṭha initiated his \emph{nyāsa} with 11 preliminary verses, concluding with praise for Sultan Hassan Shah, the son of Haider Shah, under whose reign he composed the \emph{nyāsa}.

\begin{quote}	
	\textsanskrit{grāme grāme'grahārān maṭhadharaṇiyutān karmaṭhebhyaḥ kaṭhebhyaḥ\\
	sabhyebhyo yo vyatārīdripugurunagarīryo garīyānabhaitsīt|\\
	akṣāṇyakṣīṇaśaktirvyavahṛticaturo yo vyajaiṣīcca tasmin\\
	bhūjānau \textbf{hassanā}khye bhuvamavati mayā tanyate grantha eṣaḥ|| 10||\\
	sarvakṣmāpatimaulirāhitanavaprājyasvarājyaḥ paraṃ\\
	tattvātattvavicārakāridhiṣaṇo gāmbhīryaśauryānvitaḥ |\\	
	kaśmīrādhipatiḥ kṛpājalanidhirvikhyātakīrtiściraṃ\\
	jīyād \textbf{haidaraśāhisūnu}ranaghaḥ kandarpadarpāpahaḥ || 11 ||}
	BBP 1.1.1.1.10--11
	

	“I am crafting this book during the reign of the king called \textbf{Hassana (Ḥasan)}, who bestowed royal land grants with monasteries in numerous villages to efficient courtiers, who were Brahmins of the Vedic Kaṭha school. He, the mighty one, shattered the big cities of his enemies. With ingenuity in his demeanor, he triumphed over his senses, wielding unwaning power. He stands as the pinnacle among all monarchs, ruling over his expansive new kingdom. Utmost in intelligence, he discerns between real and unreal, and is endowed with both dignity and valour. May the ruler of Kashmir, the faultless son of \textbf{Haidaraśāhi (Ḥaydar Šāh)}, an ocean of compassion, renowned for his fame, the vanquisher of Cupid's pride, live long.”	
\end{quote}

It is clear from this reference that Śitikaṇṭha began composing his work when Ḥasan Šāh (\emph{r.} 1472--1484)\footcite[109]{Hasan2024} was ruling. 

Śitikaṇṭha mentioned the exact date of the completion of the \emph{Nāmaprakaraṇa} (2nd book of \emph{Kātantra}) at the end of the first half (\emph{pūrvārdha}) of his \emph{\emph{nyāsa}}---

\begin{quote}
	

\textsanskrit{tryaṅkaviśvamite śāke varṣe muniyugairmite|\\
nabhasyasitasaptamyāṃ samāpto nāmanirṇayaḥ||}\footnote{\gls{B3} f.224v.}

“The \emph{Nāmaprakaraṇa} is completed in the Śaka year (1393) counted by three, number (=9), and \emph{Viśva} (=13) and in the Laukika year (47) counted by \emph{muni} (=7) and \emph{yuga} (4), on the seventh day of the waxing lunar fortnight (\emph{sitasaptamī}) in the month of Nabhasya (Bhādrapada).”

\end{quote}
The Laukika era, also known as the Śaptarṣi or Śāstra era, stands out as the most prevalent dating system used in the Sharada manuscripts. A significant limitation of this dating system lies in its omission of centuries from the years indicated, restarting at 0, once the count reaches 100 years. The Laukika year mentioned in manuscripts is usually the current year, while the Śaka year is most often the expired year. In this context, the Śaka year 1393 represents the expired year. The current Laukika year 47 corresponds to the current Śaka year 1394. Adhering to the Āmanta scheme, the seventh day of the waxing lunar fortnight in the month of Bhādrapada of Śaka 1394 equates to Tuesday August 20, 1472.\footnote{For the conversion, I used the free web application HIC (\url{https://hic.efeo.fr/}), originally authored by L. Gislén and J. C. Eade, developed into a web application by Toni Kustiana under the supervision of Arlo Griffiths with the funding furnished by the École française d’Extrême-Orient (EFEO).}
This specific date coincides with date of Sultan Ḥasan Šāh who ascended the throne after his father Sultan Ḥaydar Šāh's death on April 13, 1472.\footcite[108]{Hasan2024} 

In one of his prefatory verses, Śitikaṇṭha also mentions his familial ties with Bhaṭṭa Jagaddhara---

\begin{quote}

\textsanskrit{yo bālabodhinyabhidhāṃ budhendro jagaddharo yāṃ vitatāna vṛttim|\\
tannaptṛkanyātanayātanūjo vyākhyāmi tāṃ śrīśitikaṇṭhako'lpam|| 8 ||\\}

“The great intellectual Jagaddhara composed the commentary called the \emph{Bālabodhinī}. I, Śitikaṇṭha, his grandson's daughter's daughter's son, am explaining that modestly.” 

\end{quote}
Śitikaṇṭha's lineage places him five generations removed from Jagaddhara. Assuming a generational span of 20 years, we can infer that Jagaddhara flourished during the latter half of the fourteenth century.\footcite[3]{Durgaprasad1891} 

\section{Jagaddhara's Works}

The most popular works by Bhaṭṭa Jagaddhara are the \emph{Stutikusumāñjali} and the \emph{Bālabodhinī}. A significant number of manuscripts of these two texts in various archives prove their popularity. Other than these two texts, Jagaddhara also authored a few other works such as the \emph{Varṇaśikṣāsaṅkṣepa} and perhaps a \emph{citrakāvya}. 

\subsection{Stutikusumāñjali}

The \emph{Stutikusumāñjali} has remained Jagaddhara’s only published work. It is a collection of devotional hymns dedicated to Śiva spanning 1439 verses divided into 38 cantos followed by a description of author's lineage. Each of these cantos are called \emph{stotra}. Some of these \emph{stotra}s such as the \emph{Dīnākrandanastotra} or the \emph{Mahāyamakastotra}, also exist independently in separate manuscripts. The final canto of this text offerse a description of the author's lineage (see the following section for details). 

In 1681 AD, Rājānaka Ratnakaṇṭha composed a commentary on the \emph{Stutikusumāñjali} called the Laghupañcikā. 

The \emph{Stutikusumāñjali} was so popular that a number of verses from this text appeared in later anthologies. Manuscripts of the \emph{Stutikusumāñjali} in both Sharada and Devanagari are found in archives of Jammu and Kashmir as well as of Rajasthan, Maharashtra and West Bengal. A number of these manuscripts are accompanied by Ratnakaṇṭha's \emph{Laghupañcikā}.\footcite[87]{Stainton2019} A number of printed editions and translations also endorse the popularity of this text. 

The editio princeps of the \textit{Stutikusumāñjali} along with Ratnakaṇṭha's \emph{Laghupañcikā} based on two manuscripts appeared in the Kāvyamālā Series in 1891. Several other subsequent editions appeared after that until recently. However, no new manuscripts were consulted for the new editions and they all copy the text and the commentary from the editio princeps. The text has been translated into Hindi by Tripathi\footcite{Panta1964} and Sastri. Various studies on it have also been carried out.\footnote{For example, \cites{Stainton2016}{Tirtha1989}.} 


\subsection{Bālabodhinī}

While this dissertation is perhaps the effort to publish any segment of the \emph{Bālabodhinī}, the text has maintained a significant presence in the Kashmir valley. Alongside the \emph{Laghuvṛtti}, the \emph{Bālabodhinī} stands as one of the important commentaries on the \emph{Kātantra} extensively studied in Kashmir. Although \emph{Laghuvṛtti} surpasses the \emph{Bālabodhinī} in terms of the number of manuscripts found across various archives, the extant manuscripts of the Bālabodhinī are substantial. We are acquainted with over 35 manuscripts, whether complete or fragmented, of the \emph{Bālabodhinī}. Some manuscripts of the \emph{Bālabodhinī} are also available in the Devanagari script.  

Jagaddhara mentioned in one of the prefatory verses of the \emph{Bālabodhinī} that he composed the text for the study of his son Yaśodhara—

\begin{quote}
	
\textsanskrit{svasutasya śiśor yaśodharasya smaraṇārthaṃ vihito mayā śramo'yam|\\
	upayogam iyād yadi prasaṅgād aparatrāpi tato bhaved avandhyaḥ||}

“I made this effort for my own little son Yaśodhara's memorization. if it incidentally finds use elsewhere too it would not be fruitless.”
	
\end{quote}


In terms of the contents, Bālabodhinī appears to be a slightly extended version of the Laghuvṛtti. 


\subsection{Varṇaśikṣāsaṅkṣepa}

\subsection{Citrakāvya}

\subsection{Other Works Attributed to Jagaddhara}

\subsection{Śitikaṇṭha's Works}

\subsubsection{Bālabodhinīprakāśa}

